\section{Introduction}

Brief summary of measured performance, refer to ``LSST Observatory System Operations Readiness Report''.

This paper focuses on the performance of the as-built LSST system both on the compnent level and on the system level,
and compare the modeled system performance to the measured performance of the as-built system

\section{Understanding System Performance on the Component Level}

Use the as-built component information as input

Integrated modeling for capturing technical details

Results will be reconciled with system-level measurements

\subsection{General Approach for System Verification and Performance Tracking}

The LSST project has continuously maintained its best estimates of the system performance.

In early phases of the project, we had to rely heavily on simulations to baseline LSST performance.

As construction progresses, hardware is delivered, more and more as-built information and fidelity are
incorporated into the the performance baseline.
Optical figures, coating throughput, sensor noise and diffusion

The running and evolution of our performance modeling frame is an integral part of our verification process.

MagicDraw, JIRA and GitHub (with Python code and Jupyter notebooks).

\subsection{Integrated Modeling}

PSE maintains an updated model of the full system, with information from subsystems, simulating and analyzing end-to-end system performance (optics, structure and control).

It provides a key tool for non-conformance analysis, trade studies, and performance tracking.

As-built information are included as available, so that the models evolve toward the real LSST.

For system optimization: understanding system performance from the component level helps us better optimize the as-built system.

\subsection{Reconciliation between Modeled and Measured Performance}
Eventually, as we commission the integrated system, reconciliation of disagreements between measured performance
and modeled performance is a necessary part of understanding and verifying the as-built LSST,
and part of understanding sources of systematic that affect the LSST science.

\section{Key System Performance Metrics}

The LSST Science Requirement Document (SRD) is Captured by

Image quality (0.72” delivered FWHM)

Coadded depth (r = 27.5 mag, other bands sufficiently deep too)

Total survey area (18000 deg2 of sky)

Number of visits (825 times, summed over 6 bands)

Requirements flow-down

Not a comprehensive list

\subsection{Integrated \'etendue}

Definition, 3 f-factors

\subsubsection{The Fill Factor $f_F$}

Definition, as-built value

\subsubsection{The Sensitivity Factor $f_S$}

Definition, m5 results, m5 by amplifier

\subsubsection{The Observing Efficiency Factor $f_O$}

Definition. Briefly summarize OpSim \cite{2018arXiv181004815N}. Results

\subsection{Image Quality}

IQ budget tree status. IQ verification. Compared aggregated image quality to measured image quality.

\subsection{Throughput}

latest curves. Compare aggregated throughput curves to measured throughput. Refer to
``Performance of the LSST Calibration Systems''.

\subsection{Other Metrics}

\section{Mitigation and Optimization}

Here are a couple of examples of using the “f-metrics” to perform survey mitigation and optimization.

\section{Summary and Conclusions}

Recap system image quality.

Assuming fiducial atmosphere of 0.6 arcsec FWHM, delivered IQ by band.

Integrated \'etendue prediction.
