\section{Introduction}

This section provides a brief summary of measured performance. Readers are referred to \cite{PSTN-004} for more details.

This paper focuses on the performance of the as-built LSST system both on the component level and on the system level, and compare the modeled system performance to the measured performance of the as-built system

\section{Understanding System Performance on the Component Level}

Use the as-built component information as input

Integrated modeling for capturing technical details

Results are reconciled with system-level measurements

We also talk about our general approach for system verification and performance tracking

% The LSST project has continuously maintained its best estimates of the system performance.

% In early phases of the project, we had to rely heavily on simulations to baseline LSST performance.

% As construction progresses, hardware is delivered, more and more as-built information and fidelity are
% incorporated into the the performance baseline.
% Optical figures, coating throughput, sensor noise and diffusion

% The running and evolution of our performance modeling frame is an integral part of our verification process.

% MagicDraw, JIRA and GitHub (with Python code and Jupyter notebooks).

Integrated Modeling

% PSE maintains an updated model of the full system, with information from subsystems,
% simulating and analyzing end-to-end system performance (optics, structure and control).
% PhoSim as the optical engine~\cite{2015ApJS..218...14P}.

% It provides a key tool for non-conformance analysis, trade studies, and performance tracking.

% As-built information are included as available, so that the models evolve toward the real LSST.

% For system optimization: understanding system performance from the component level helps us better optimize the as-built system.

and the reconciliation between modeled and measured performance.
% Eventually, as we commission the integrated system, reconciliation of disagreements between measured performance
% and modeled performance is a necessary part of understanding and verifying the as-built LSST,
% and part of understanding sources of systematic that affect the LSST science.

\section{Key System Performance Metrics}

The LSST Science Requirement Document (SRD) is Captured by

\begin{itemize}
\item Image quality (0.72 arcsec delivered FWHM)

\item Coadded depth (r = 27.5 mag, other bands sufficiently deep too)

\item Total survey area (18000 deg$^2$ of sky)

\item Number of visits (825 times, summed over 6 bands)
\end{itemize}

Note that this is not a comprehensive list

We also talk about requirements flow-down here.



\subsection{Integrated \'etendue}

Definition, 3 f-factors

\subsubsection{The Fill Factor $f_F$}

Definition, as-built value

\subsubsection{The Sensitivity Factor $f_S$}

Definition, m5 results, m5 by amplifier.

Refer to \cite{2016SPIE.9910E..1AY} for Sky brightness model.

\subsubsection{The Observing Efficiency Factor $f_O$}

Definition. Briefly summarize OpSim ~\cite{PSTN-007} and \cite{PSTN-043}

\subsection{Image Quality}

IQ budget tree. IQ verification. Compared aggregated image quality to measured image quality.

Assuming fiducial atmosphere of 0.6 arcsec FWHM, delivered IQ by band.
Also give IQ results using our current best knowledge about Cerro Pachon atmosphere.
The former is used for performance characterization, and the latter is used in OpSim simulations.

\subsection{Throughput}

latest curves. Compare aggregated throughput curves to measured throughput. Refer to~\cite{PSTN-027}.

\subsection{Other Metrics}

\section{Integrated Modeling}

PSE maintains an updated model of the full system, with information from subsystems,
simulating and analyzing end-to-end system performance (optics, structure and control).
Refer to past SPIE papers \cite{2016SPIE.9911E..18A, 2018SPIE10705E..0PX}.

Show Integrated Model block diagram.

PhoSim as the optical engine~\cite{2015ApJS..218...14P}.

% It provides a key tool for non-conformance analysis, trade studies, and performance tracking.

% As-built information are included as available, so that the models evolve toward the real LSST.

% For system optimization: understanding system performance from the component level helps us better optimize the as-built system.

\section{Mitigation and Optimization}

Give examples/list of using the Integrated Model for non-conformance analysis and trade studies.

Here are a couple of examples of using the “f-metrics” to perform survey mitigation and optimization.

\section{Summary and Conclusions}

Recap system image quality and m5 performance.

Assuming fiducial atmosphere of 0.6 arcsec FWHM, delivered IQ by band.

Integrated \'etendue prediction.
